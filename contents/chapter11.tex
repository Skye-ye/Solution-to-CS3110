\section*{Chapter 11. Interpreters}
\addcontentsline{toc}{section}{Chapter 11. Interpreters}

\problem[pair]
Add pairs (i.e., tuples with exactly two components) to SimPL.

\begin{lstlisting}[language=OCaml]
(* ast.ml *)

(** The type of binary operators. *)
type bop = 
  | Add
  | Mult
  | Leq

(** The type of the abstract syntax tree (AST). *)
type expr =
  | Var of string
  | Int of int
  | Bool of bool  
  | Binop of bop * expr * expr
  | Let of string * expr * expr
  | If of expr * expr * expr
  | Pair of expr * expr

(* lexer.mll *)

{
open Parser
}

let white = [' ' '\t']+
let digit = ['0'-'9']
let int = '-'? digit+
let letter = ['a'-'z' 'A'-'Z']
let id = letter+

rule read = 
  parse
  | white { read lexbuf }
  | "true" { TRUE }
  | "false" { FALSE }
  | "<=" { LEQ }
  | "*" { TIMES }
  | "+" { PLUS }
  | "(" { LPAREN }
  | ")" { RPAREN }
  | "let" { LET }
  | "=" { EQUALS }
  | "in" { IN }
  | "if" { IF }
  | "then" { THEN }
  | "else" { ELSE }
  | "," { COMMA }
  | id { ID (Lexing.lexeme lexbuf) }
  | int { INT (int_of_string (Lexing.lexeme lexbuf)) }
  | eof { EOF }

(* parser.mly *)

%{
open Ast
%}

%token <int> INT
%token <string> ID
%token TRUE
%token FALSE
%token LEQ
%token TIMES  
%token PLUS
%token LPAREN
%token RPAREN
%token LET
%token EQUALS
%token IN
%token IF
%token THEN
%token ELSE
%token COMMA
%token EOF

%nonassoc IN
%nonassoc ELSE
%left LEQ
%left PLUS
%left TIMES  

%start <Ast.expr> prog

%%

prog:
	| e = expr; EOF { e }
	;
	
expr:
	| i = INT { Int i }
	| x = ID { Var x }
	| TRUE { Bool true }
	| FALSE { Bool false }
	| e1 = expr; LEQ; e2 = expr { Binop (Leq, e1, e2) }
	| e1 = expr; TIMES; e2 = expr { Binop (Mult, e1, e2) } 
	| e1 = expr; PLUS; e2 = expr { Binop (Add, e1, e2) }
	| LET; x = ID; EQUALS; e1 = expr; IN; e2 = expr { Let (x, e1, e2) }
	| IF; e1 = expr; THEN; e2 = expr; ELSE; e3 = expr { If (e1, e2, e3) }
	| LPAREN; e1 = expr; COMMA; e2 = expr; RPAREN { Pair (e1, e2) }
	| LPAREN; e=expr; RPAREN {e} 
	;


(* main.ml *)

open Ast

(** [parse s] parses [s] into an AST. *)
let parse (s : string) : expr =
  let lexbuf = Lexing.from_string s in
  let ast = Parser.prog Lexer.read lexbuf in
  ast

(** [typ] represents the type of an expression. *)
type typ =
  | TInt
  | TBool
  | TPair of typ * typ

(** The error message produced if a variable is unbound. *)
let unbound_var_err = "Unbound variable"

(** The error message produced if binary operators and their operands do not
    have the correct types. *)
let bop_err = "Operator and operand type mismatch"

(** The error message produced if the [then] and [else] branches of an [if] do
    not have the same type. *)
let if_branch_err = "Branches of if must have same type"

(** The error message produced if the guard of an [if] does not have type
    [bool]. *)
let if_guard_err = "Guard of if must have type bool"

(** A [Context] is a mapping from variable names to types, aka a symbol table,
    aka a typing environment. *)
module type Context = sig
  (** [t] is the type of a context. *)
  type t

  (** [empty] is the empty context. *)
  val empty : t

  (** [lookup ctx x] gets the binding of [x] in [ctx]. Raises:
      [Failure unbound_var_err] if [x] is not bound in [ctx]. *)
  val lookup : t -> string -> typ

  (** [extend ctx x ty] is [ctx] extended with a binding of [x] to [ty]. *)
  val extend : t -> string -> typ -> t
end

(** The [Context] module implements the [Context] signature with an association
    list. *)
module Context : Context = struct
  type t = (string * typ) list

  let empty = []

  let lookup ctx x =
    try List.assoc x ctx with Not_found -> failwith unbound_var_err

  let extend ctx x ty = (x, ty) :: ctx
end

open Context

(** [typeof ctx e] is the type of [e] in context [ctx]. Raises: [Failure] if [e]
    is not well typed in [ctx]. *)
let rec typeof ctx = function
  | Int _ -> TInt
  | Bool _ -> TBool
  | Var x -> lookup ctx x
  | Let (x, e1, e2) -> typeof_let ctx x e1 e2
  | Binop (bop, e1, e2) -> typeof_bop ctx bop e1 e2
  | If (e1, e2, e3) -> typeof_if ctx e1 e2 e3
  | Pair (e1, e2) -> typeof_pair ctx e1 e2

(** Helper function for [typeof]. *)
and typeof_let ctx x e1 e2 =
  let t1 = typeof ctx e1 in
  let ctx' = extend ctx x t1 in
  typeof ctx' e2

(** Helper function for [typeof]. *)
and typeof_bop ctx bop e1 e2 =
  let t1, t2 = (typeof ctx e1, typeof ctx e2) in
  match (bop, t1, t2) with
  | Add, TInt, TInt | Mult, TInt, TInt -> TInt
  | Leq, TInt, TInt -> TBool
  | _ -> failwith bop_err

(** Helper function for [typeof]. *)
and typeof_if ctx e1 e2 e3 =
  if typeof ctx e1 = TBool then begin
    let t2 = typeof ctx e2 in
    if t2 = typeof ctx e3 then t2 else failwith if_branch_err
  end
  else failwith if_guard_err

(** Helper function for [typeof]. *)
and typeof_pair ctx e1 e2 =
  let t1, t2 = (typeof ctx e1, typeof ctx e2) in
  TPair (t1, t2)

(** [typecheck e] checks whether [e] is well typed in the empty context. Raises:
    [Failure] if not. *)
let typecheck e = ignore (typeof empty e)

(** [is_value e] is whether [e] is a value. *)
let rec is_value : expr -> bool = function
  | Int _ | Bool _ -> true
  | Var _ | Let _ | Binop _ | If _ -> false
  | Pair (e1, e2) -> is_value e1 && is_value e2

(** [subst e v x] is [e] with [v] substituted for [x], that is, [e{v/x}]. *)
let rec subst e v x =
  match e with
  | Var y -> if x = y then v else e
  | Bool _ -> e
  | Int _ -> e
  | Binop (bop, e1, e2) -> Binop (bop, subst e1 v x, subst e2 v x)
  | Let (y, e1, e2) ->
    let e1' = subst e1 v x in
    if x = y then Let (y, e1', e2) else Let (y, e1', subst e2 v x)
  | If (e1, e2, e3) -> If (subst e1 v x, subst e2 v x, subst e3 v x)
  | Pair (e1, e2) -> Pair (subst e1 v x, subst e2 v x)

(** [step] is the [-->] relation, that is, a single step of evaluation. *)
let rec step : expr -> expr = function
  | Int _ | Bool _ -> failwith "Does not step"
  | Var _ -> failwith unbound_var_err
  | Binop (bop, e1, e2) when is_value e1 && is_value e2 -> step_bop bop e1 e2
  | Binop (bop, e1, e2) when is_value e1 -> Binop (bop, e1, step e2)
  | Binop (bop, e1, e2) -> Binop (bop, step e1, e2)
  | Let (x, e1, e2) when is_value e1 -> subst e2 e1 x
  | Let (x, e1, e2) -> Let (x, step e1, e2)
  | If (Bool true, e2, _) -> e2
  | If (Bool false, _, e3) -> e3
  | If (Int _, _, _) -> failwith if_guard_err
  | If (e1, e2, e3) -> If (step e1, e2, e3)
  | Pair (e1, e2) when is_value e1 -> Pair (e1, step e2)
  | Pair (e1, e2) -> Pair (step e1, e2)

(** [step_bop bop v1 v2] implements the primitive operation [v1 bop v2].
    Requires: [v1] and [v2] are both values. *)
and step_bop bop e1 e2 =
  match (bop, e1, e2) with
  | Add, Int a, Int b -> Int (a + b)
  | Mult, Int a, Int b -> Int (a * b)
  | Leq, Int a, Int b -> Bool (a <= b)
  | _ -> failwith bop_err

(** [eval_small e] is the [e -->* v] relation. That is, keep applying [step]
    until a value is produced. *)
let rec eval_small (e : expr) : expr =
  if is_value e then e else e |> step |> eval_small

(** [interp_small s] interprets [s] by parsing, type-checking, and evaluating it
    with the small-step model. *)
let interp_small (s : string) : expr =
  let e = parse s in
  typecheck e;
  eval_small e

(** [eval_big e] is the [e ==> v] relation. *)
let rec eval_big (e : expr) : expr =
  match e with
  | Int _ | Bool _ -> e
  | Var _ -> failwith unbound_var_err
  | Binop (bop, e1, e2) -> eval_bop bop e1 e2
  | Let (x, e1, e2) -> subst e2 (eval_big e1) x |> eval_big
  | If (e1, e2, e3) -> eval_if e1 e2 e3
  | Pair (e1, e2) -> Pair (eval_big e1, eval_big e2)

(** [eval_bop bop e1 e2] is the [e] such that [e1 bop e2 ==> e]. *)
and eval_bop bop e1 e2 =
  match (bop, eval_big e1, eval_big e2) with
  | Add, Int a, Int b -> Int (a + b)
  | Mult, Int a, Int b -> Int (a * b)
  | Leq, Int a, Int b -> Bool (a <= b)
  | _ -> failwith bop_err

(** [eval_if e1 e2 e3] is the [e] such that [if e1 then e2 else e3 ==> e]. *)
and eval_if e1 e2 e3 =
  match eval_big e1 with
  | Bool true -> eval_big e2
  | Bool false -> eval_big e3
  | _ -> failwith if_guard_err

(** [interp_big s] interprets [s] by parsing, type-checking, and evaluating it
    with the big-step model. *)
let interp_big (s : string) : expr =
  let e = parse s in
  typecheck e;
  eval_big e

\end{lstlisting}