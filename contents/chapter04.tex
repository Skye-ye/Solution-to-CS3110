\section*{Chapter 4. The Basics of OCaml}
\addcontentsline{toc}{section}{Chapter 4. The Basics of OCaml}

\problem[date fun]
Define a function that takes an integer \B{d} and string \B{m} as input and returns \B{true} just when \B{d} and \B{m} form a valid date. Here, a valid date has a month that is one of the following abbreviations: Jan, Feb, Mar, Apr, May, Jun, Jul, Aug, Sept, Oct, Nov, Dec. The day must be a number that is between 1 and the minimum number of days in that month, inclusive. For example, if the month is Jan, then the day is between 1 and 31, inclusive, whereas if the month is Feb, then the day is between 1 and 28, inclusive. How terse (i.e., few and short lines of code) can you make your function? You can definitely do this in fewer than 12 lines.

\begin{lstlisting}[language=OCaml]
let valid_date (d : int) (m : string) =
  let max_days =
    match m with
      | "Jan" | "Mar" | "May" | "Jul" | "Aug" | "Oct" | "Dec" -> 31
      | "Apr" | "Jun" | "Sept" | "Nov" -> 30
      | "Feb" -> 28
      | _ -> 0 (* Invalid month abbreviation *)
  in
  d >= 1 && d <= max_days
\end{lstlisting}

\problem[fib]
Define a recursive function \code{fib : int -> int}, such that \code{fib n} is the $n$th number in the Fibonacci sequence,
which is $1, 1, 2, 3, 5, 8, 13, \ldots$ That is:
\begin{itemize}
  \item \code{fib 1 = 1}
  \item \code{fib 2 = 1}
  \item \code{fib n = fib (n - 1) + fib (n - 2)} for any $n > 2$
\end{itemize}

\begin{lstlisting}[language=OCaml]
let rec fib n =
  if n = 0 then 0
  else if n = 1 then 1
  else fib (n - 1) + fib (n - 2)
\end{lstlisting}

\problem[fib fast]
Write a tail-recursive version of \code{fib}.

\begin{lstlisting}[language=OCaml]
let fib n =
  let rec fib_tail n a b =
    if n = 0 then a
    else fib_tail (n - 1) b (a + b)
  in
  if n <= 0 then invalid_arg "fib: negative input"
  else fib_tail n 0 1
\end{lstlisting}

\problem[divide]
Write a function \code{divide : numerator:float -> denominator:float -> float}. Apply your function to 1.0 and 2.0.

\begin{lstlisting}[language=OCaml]
let divide ~numerator:x ~denominator:y = x /. y
\end{lstlisting}

\problem[average]
Define an infix operator \code{+/.} to compute the average of two floating-point numbers. For example,
\begin{itemize}
  \item $1.0 +/. 2.0 = 1.5$
  \item $0.0 +/. 0.0 = 0.0$
\end{itemize}

\begin{lstlisting}[language=OCaml]
let (+/.) x y = (x +. y) /. 2.0
\end{lstlisting}